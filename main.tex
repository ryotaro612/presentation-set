% unicodeは、hyperrefへの指定で、pdfのメタデータにあるタイトルの文字化けを防ぐ
% ptは細かい指定はできないらしい
% チートシート
% https://www.cpt.univ-mrs.fr/~masson/latex/Beamer-appearance-cheat-sheet.pdf
\documentclass[unicode, 14pt, aspectratio=169]{beamer} 
% \usepackage{listings}
% \usepackage{enumitem}
% \usepackage{xcolor}
% \usepackage{textcomp}
% styleはここで指定できる
% スタイルの一覧
% https://www.overleaf.com/learn/latex/Biblatex_bibliography_styles
\usetheme{rikako}
\date{\number\year 年\number\month 月\number\day 日}
\addbibresource{main.bib}
\title{型理論がうまれた背景}
\author{\texttt{ryotaro612}} 
\begin{document}
\begin{frame}[noframenumbering, plain]
\titlepage
\end{frame}
\section{導入}
\begin{frame}
  \frametitle{ねらい}
  \large
  型理論ができた背景から、近い分野に興味をもってもらう
  \normalsize
\end{frame}
\section{幾何学の起源}
\begin{frame}
\end{frame}
\section{論理学}
\section{型}
  
% https://fair-use.org/bertrand-russell/the-principles-of-mathematics/s498
% https://plato.stanford.edu/entries/type-theory/
% https://people.umass.edu/klement/pom/pom-portrait.pdf
% https://repository.kulib.kyoto-u.ac.jp/dspace/bitstream/2433/151121/1/ronso38_S49_type.pdf
% https://www.britannica.com/topic/theory-of-types-logic
\begin{frame}[t]
  \frametitle{runc}
\end{frame}
\begin{frame}[allowframebreaks,t]
  \frametitle{参考資料}  
  \printbibliography
  % \nocite{*}
\end{frame}

\end{document}
