% unicodeは、hyperrefへの指定で、pdfのメタデータにあるタイトルの文字化けを防ぐ
% ptは細かい指定はできないらしい
% チートシート
% https://www.cpt.univ-mrs.fr/~masson/latex/Beamer-appearance-cheat-sheet.pdf
\documentclass[unicode, 14pt, aspectratio=169]{beamer} 
% \usepackage{minted}
% \usepackage{listings}
% \usepackage{enumitem}
% \usepackage{xcolor}
% \usepackage{textcomp}
\usepackage[backend=biber, style=ieee]{biblatex}
\usetheme{rikako}
\date{\number\year 年\number\month 月\number\day 日}
% \addbibresource{main.bib}
\title{runcのUNIXプログラミング}
\author{\texttt{ryotaro612}}
\begin{document}
\usemintedstyle{titech}
\begin{frame}[noframenumbering, plain]
\titlepage
\end{frame}
\section{導入}
% https://fair-use.org/bertrand-russell/the-principles-of-mathematics/s498
% https://plato.stanford.edu/entries/type-theory/
% https://people.umass.edu/klement/pom/pom-portrait.pdf
% https://repository.kulib.kyoto-u.ac.jp/dspace/bitstream/2433/151121/1/ronso38_S49_type.pdf
% https://www.britannica.com/topic/theory-of-types-logic
\begin{frame}[t]
  \frametitle{runc}
\end{frame}
\begin{frame}[allowframebreaks,t]
  \frametitle{参考資料}
  \printbibliography
  % \nocite{*}
\end{frame}
\end{document}
