% unicodeは、hyperrefへの指定で、pdfのメタデータにあるタイトルの文字化けを防ぐ
% ptは細かい指定はできないらしい
% チートシート
% https://www.cpt.univ-mrs.fr/~masson/latex/Beamer-appearance-cheat-sheet.pdf
\documentclass[unicode, 14pt, aspectratio=169]{beamer}
% \usepackage{listings}
% \usepackage{enumitem}
% \usepackage{xcolor}
% \usepackage{textcomp}
% styleはここで指定できる
% スタイルの一覧
% https://www.overleaf.com/learn/latex/Biblatex_bibliography_styles
\usetheme{titech}
\date{\number\year 年\number\month 月\number\day 日}
\addbibresource{main.bib}
\title{型理論のはじまり}
\author{\texttt{ryotaro612}} 
\begin{document}
\begin{frame}[noframenumbering, plain]
\titlepage
\end{frame}
\section{導入}
\begin{frame}
  \frametitle{資料の目的}
  {\large 型理論に近い分野への関心のきっかけをつくる}
\end{frame}
\section{非ユークリッド幾何学の発見}
\begin{frame}
  \frametitle{ギリシアで発展した幾何学}
  {\large 幾何学は定理を証明することで発展した}
  定理を演繹で証明して発展させた
  公理からの演繹
  % タレス、ピタゴラス、プラトンアカデメイア、アリストテレス
  % アルキメデス 有理数
  % 球体
  % ロバチェフスキー
  % 公準
  % 素朴集合論
  % 双曲平面橋爪
\end{frame}
\begin{frame}
  \frametitle{ユークリッドの原論}
  {\large 原論は、定義、公理、公準、命題、証明を分けた}
  % ユークリッドはギリシア数学の知識を5つの公準から
\end{frame}
\begin{frame}
  \frametitle{5つの公準}
  {\large 5番目の平行線の公準はほかよりも複雑}
\end{frame}
\begin{frame}
  \frametitle{公準5の別名は平行線公理}
  {\large 任意の直線とその直線外の任意の一点があるとき、その一点を通りその直線に平行な直線は1本だけ}
\end{frame}
\begin{frame}
  \frametitle{平行線公理への疑問}
  {\large 平行線公理は必要な公準か}
  たんなる仮説なのか
  ほかの公準から導出できるか
% 直線$l$上にない任意の点$P$を通る直線$l$の平行線は2本以上ある
\end{frame}
\begin{frame}
  \frametitle{例 球面幾何学の平行線}
  {\large 測地線があたえられるとき、その測地線にない点pをとおり測地線と交わらない測地線はない}
  たんなる仮説なのか
  ほかの公準から導出できるか
\end{frame}
\begin{frame}
  \frametitle{非ユークリッド幾何学の発見の影響}
  {\large 「正しさ」は仮定した公理から作りだすもの}
\end{frame}
\section{素朴な集合論と記号論理学}
\begin{frame}
  \frametitle{ものの集まりで数学を考える}
  {\large いろんな数学の概念を集合で定義できる}
\end{frame}
\begin{frame}
  \frametitle{算術の基礎}
  {\large フレーゲは記号論理学と集合論による自然数の公理と定理の提供を試みた}
\end{frame}
\begin{frame}
  \frametitle{思考を計算するための記号}
  {\large 推論の規則を記号を操作する計算に還元する考えは世紀以前からある}
\end{frame}
\begin{frame}
  \frametitle{ブールの論理代数}
  {\large ブールは命題に出現する個体の集合を記号で表した}
  % https://archive.org/stream/THELAWSOFTHOUGHTGeorgeBoole/THE%20LAWS%20OF%20THOUGHT_George%20Boole_djvu.txt
\end{frame}
\begin{frame}
  \frametitle{推論規則の例}
  {\large 命題論理は、命題と論理結合子からなる論理体系}
\end{frame}
\begin{frame}
  \frametitle{概念記法}
  {\large フレーゲは述語の言明を形式化した}
\end{frame}
\begin{frame}
  \frametitle{ラッセルのパラドクス}
  {\large 集合の集合を考えると矛盾がおきる}
\end{frame}
\begin{frame}
  \frametitle{型の概念}
  ラッセルの自分自身に言明する述語の例に対して、フレーゲは述語は述語を引数にとれないと回答した

  ただし述語の外延が作用する
\end{frame}
% https://fair-use.org/bertrand-russell/the-principles-of-mathematics/s498
% https://plato.stanford.edu/entries/type-theory/
% https://people.umass.edu/klement/pom/pom-portrait.pdf
% https://repository.kulib.kyoto-u.ac.jp/dspace/bitstream/2433/151121/1/ronso38_S49_type.pdf
% https://www.britannica.com/topic/theory-of-types-logic
\begin{frame}[allowframebreaks,t]
  \frametitle{参考資料}  
  \printbibliography
  % \nocite{*}
\end{frame}

\end{document}
